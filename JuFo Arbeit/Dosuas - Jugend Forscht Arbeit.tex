\documentclass[a4paper,10pt,ngerman]{scrartcl}
\usepackage{babel}
\usepackage[T1]{fontenc}
\usepackage[utf8x]{inputenc}
\usepackage[a4paper,margin=2.5cm]{geometry}

\usepackage[colorlinks=true,linkcolor=blue,urlcolor=blue,
            citecolor=blue,anchorcolor=blue]{hyperref}

\title{Dosuas - Die Symphonie des Sehens}
\subtitle{Jugend Forscht 2018}
\author{Jonas Wanke und Yorick Zeschke}
\date{\today}

\begin{document}

\maketitle

\begin{abstract}
Dosuas (\textbf{D}evice for \textbf{O}rientation in \textbf{S}pace \textbf{U}sing 
\textbf{A}udio \textbf{S}ignals) ist ein Gerät, welches blinden Menschen ermöglicht
sich mithilfe von Tonsignalen im Raum zu orientieren und Objekte zu erkennen.\par
Das Projekt besteht aus zwei Unterprojekten, die beide bis zum 
Wettbewerb als Prototypen umgesetzt werden sollen. Einmal werden Bilder eines 3D-Sensors 
algorithmisch in Töne umgewandelt, die dann mit 3D-Audio Kopfhörern hörbar gemacht
werden. Die andere Idee basiert darauf, so ähnlich wie eine Fledermaus Ultraschall 
Impulse zu senden und deren Reflektionen bzw. Echos hörbar zu machen, sodass man
sich mit Klicklauten orientieren kann. Letzteres basiert auf der Technik der aktiven
\href{https://de.wikipedia.org/wiki/Menschliche_Echoortung}{menschlichen Echoortung}.
\end{abstract}

\tableofcontents

\end{document}