\usepackage{tikz}
\usetikzlibrary{arrows.meta}
\usepackage{forest}
\forestset{qtree/.style={for tree={align=center, grow'=right, edge={green!50!black, -Latex}}}}


\begin{forest}
	qtree
	[{Start}, name=start, s sep=0.5cm
	[{automatisierte Echoortung\\ via Ultraschall}
	[{aufw�ndige Entwicklung,\\ schlechtere Qualit�t als\\ menschliche Echoortung,\\ langer Lernprozess}, name=us_end, edge={red}]]
	[{Modellierung gro�er\\ Objekte als Tonquellen}
	[{zu wenig Informationen,\\ schlechte Orientierung}, name=objects_end, edge={red}]]
	[, no edge]
	[, no edge]
	[, no edge]
	[{2D Radar-Swipe\\ (jetziger Anf�nger-\\Modus)}
	[{Verbesserter Anf�nger-Modus\\ (beim Regionalwettbewerb vorgestellt)}]
	[{Fortgeschrittenen-\\Modus}
	[{siehe Ausblick}]]]]
	\useasboundingbox ([xshift=1cm, yshift=2.5cm]current bounding box.north west) rectangle (current bounding box.south east);
	\draw [green!50!black, -Latex] (us_end.north west) to [out=160, in=60] node[midway, above, black] {Abbruch nach Regionalwettbewerb} (0.2cm, 3.5cm) to [out=240, in=100] (start.north);
	\draw [green!50!black, -Latex] (objects_end.south west) to [bend left=20] node[midway, below, xshift=1.5cm, black] {Abbruch nach Regionalwettbewerb} (start.south east);
\end{forest}